\documentclass[]{article}
\begin{document}

\section{Meter Output}
The meter outputs a comma-delimited packet of values.

\begin{tabular}{ c l c l l }
Order & Title          & Units     & Code Variable  & ADE Variable \\
0     & uCTime         &           &                & ---          \\
1     & sequenceNumber &           &                & ---          \\
2     & circuitID      &           &                & ---          \\
3     & isOn           &           &                & ---          \\
4     & VRMS           &           & c--$>$VRMS     &              \\
5     & IRMS           &           & c--$>$IRMS     &              \\
6     & VPEAK          &           & c--$>$vpeak    & RSTVPEAK     \\
7     & IPEAK          &           & c--$>$ipeak    & RSTIPEAK     \\
8     & PERIODUS       &           & c--$>$periodus &              \\
9     & VA             &           & c--$>$VA       & LVAENERGY    \\
10    & W              &           & c--$>$W        & LAENERGY     \\
11    & VAEnergy       & Joules?   & c--$>$VAEnergy & RVAENERGY     \\
12    & WEnergy        & Joules?   & c--$>$WEnergy  & RAENERGY     \\
13    & PFactor        &           & c--$>$PF       &              \\
14    & VA ACCUM       &           & TODO        \\
15    & W ACCUM        &           & TODO        \\
16    & errorNum       &           &             \\
\end{tabular}

\section{Measurements}
This section details the measurement of calibrated loads and the
measured values resulting.

\subsection{Maximum Load}

\subsection{Minimum Detectable Load}

\subsection{Power Factor Measurements}

\section{Meter DC Consumption}
Per meter marginal increase: 50mA @ 5V, .25W


\subsection{Motherboard DC Consumption}
Without any meters: <=90mA, .5W
Complete system with 20 meters: 1.1A @ 5V, 5.5W


\subsubsection{Measurement Method}
\subsubsection{Measurement Result}

\subsection{Daughterboard DC Consumption}

\subsubsection{Measurement Method}
\subsubsection{Measurement Result}
\end{document}
