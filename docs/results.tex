\documentclass[]{article}
\begin{document}

\section{Meter Output}
The meter outputs a comma-delimited packet of values.

\begin{tabular}{ c l c l l }
Order & Title          & Units     & Code Variable  & ADE Variable \\
0     & uCTime         &           &                & ---          \\
1     & sequenceNumber &           &                & ---          \\
2     & circuitID      &           &                & ---          \\
3     & isOn           &           &                & ---          \\
4     & VRMS           &           & c--$>$VRMS     &              \\
5     & IRMS           &           & c--$>$IRMS     &              \\
6     & VPEAK          &           & c--$>$vpeak    & RSTVPEAK     \\
7     & IPEAK          &           & c--$>$ipeak    & RSTIPEAK     \\
8     & PERIODUS       &           & c--$>$periodus &              \\
9     & VA             &           & c--$>$VA       & LVAENERGY    \\
10    & W              &           & c--$>$W        & LAENERGY     \\
11    & VAEnergy       & Joules?   & c--$>$VAEnergy & RVAENERGY     \\
12    & WEnergy        & Joules?   & c--$>$WEnergy  & RAENERGY     \\
13    & PFactor        &           & c--$>$PF       &              \\
14    & VA ACCUM       &           & TODO        \\
15    & W ACCUM        &           & TODO        \\
16    & errorNum       &           &             \\
\end{tabular}

\section{Measurements}
This section details the measurement of calibrated loads and the
measured values resulting.

\subsection{Maximum Load}

\subsection{Minimum Detectable Load}

\subsection{Power Factor Measurements}

\section{Meter DC Consumption}
Per meter marginal increase: 50mA @ 5V, .25W

\subsection{Motherboard DC Consumption}

\subsubsection{Measurement Method}
Measurements were taken directly from the power supply.
\subsubsection{Measurement Result}
Without any meters: <=90mA, .5W \\*
Complete system with 20 meters: 1.1A @ 5V, 5.5W

\subsection{Daughterboard DC Consumption}
Per meter increase: 50mA @ 5V, .25W
For two daughterboards: 100mA @ 5V, .5W

\subsubsection{Measurement Method}
Measurements were taken directly from the power supply.
\subsubsection{Measurement Result}

\section{Testing Protocol}
\subsection{Functionality}
\subsection{Noise}
\subsection{Software Tests}

\section{Verification}
    \subsection{Telduino}
        \subsubsection{Inspection}
            Solder bridges on the atmega micro-controller
            Solder bridges or incomplete connections on the FTDI usb-serial converter
            Reversed polarity of capacitors see layout for correct polarity
                For power circuit the output voltage may slowly drop if any of the polarized capacitors are in reverse and you are using the linear regulator.
                This is also true of the polarized capacitors near the modem and the atmega.
                If an AC/DC adapter is used without the ability to detect a short, the capacitors may ignite.
                The capacitors near the crystal are very close.

        \subsubsection{Testing}
            * Ensure jumpers are disconnected
                -- which jumpers
            Test
            Expected
            Error
            GSM
            Test for GSM serial communications
            Type AT at the terminal will recieve OK back.
            Test antenna basic functionality
            Send a quick SMS using the commands
            Expected to recieve in x seconds

            Test correct power voltage
            Conditions
            Expect 3.6V across pins x and y
            Maximum load is 2A put X ohm resistor across pins A and B and ensure that 3.6V is still present

            Test USB to serial chip
            Ensure connection to computer with commands ...
            Probe pins X and Y while typing
            Jump pins to do echo?

            Test Micro Controller

            Test Power Draw

    \subsection{Meter Daughterboard}
        * Ensure jumpers are disconnected
             -- which jumpers
        \subsubsection{Inspection}
            The DC/DC converter can easily be unseated and put out of alignment. Check the leads for colds solders.
            The Optoisolators can be easily cold soldered as well.
            Check the crystals around the ADEs for solder bridges. The capacitors are very close.
        \subsubsection{Testing}

            Integrate notes from docs
            TODO add female headers to parts list for completeness
            Find a telit source.
            Test Power Draw
    \subsection{System}
    \subsubsection{Noise}
        Real world loads
        CFL
        Fan
        Power charger
        Switching test

    \subsubsection{Metering Capability}
    Test small load
    Test minimum visible
    Test minimum measureable
    Test maximum
    Test large load
    Test reactive load
    Test passive load
    Test switch functionality
    \subsubsection{Ratings}
    Maximum current draw
    Maximum voltage

    \subsection{ToFix}
        See: https://github.com/modilabs/telduino/issues/
\end{document}

