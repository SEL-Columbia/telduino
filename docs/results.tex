\documentclass[]{article}
\begin{document}

\section{Meter Output}
The meter outputs a comma-delimited packet of values.

\begin{tabular}{ c l c l l }
Order & Title          & Units     & Code Variable  & ADE Variable \\
0     & uCTime         &           &                & ---          \\
1     & sequenceNumber &           &                & ---          \\
2     & circuitID      &           &                & ---          \\
3     & isOn           &           &                & ---          \\
4     & VRMS           &           & c--$>$VRMS     &              \\
5     & IRMS           &           & c--$>$IRMS     &              \\
6     & VPEAK          &           & c--$>$vpeak    & RSTVPEAK     \\
7     & IPEAK          &           & c--$>$ipeak    & RSTIPEAK     \\
8     & PERIODUS       &           & c--$>$periodus &              \\
9     & VA             &           & c--$>$VA       & LVAENERGY    \\
10    & W              &           & c--$>$W        & LAENERGY     \\
11    & VAEnergy       & Joules?   & c--$>$VAEnergy & RVAENERGY     \\
12    & WEnergy        & Joules?   & c--$>$WEnergy  & RAENERGY     \\
13    & PFactor        &           & c--$>$PF       &              \\
14    & VA ACCUM       &           & TODO        \\
15    & W ACCUM        &           & TODO        \\
16    & errorNum       &           &             \\
\end{tabular}

\section{Measurements}
This section details the measurement of calibrated loads and the
measured values resulting.

\subsection{Maximum Load}
Imax = 11.5A
Vmax = 250VAC
The daughterboards voltage limits is determined by the relays and spacing of traces to be 250VAC.
The current rating is limited to 16A or the current sense resistor power dissipation. Presently, that is 4W so Imax = sqrt(4W/.03 Ohms)

\subsection{Minimum Detectable Load}

\subsection{Power Factor Measurements}

\section{Meter DC Consumption}
    Per meter marginal increase: 50mA @ 5V, .25W

\subsection{Motherboard DC Consumption}

\subsubsection{Measurement Method}
    Measurements were taken directly from the power supply.
\subsubsection{Measurement Result}
    Without any meters: <=90mA, .5W \\*
    Complete system with 20 meters: 1.1A @ 5V, 5.5W

\subsection{Daughterboard DC Consumption}
    Per meter increase: 50mA @ 5V, .25W
    For two daughterboards: 100mA @ 5V, .5W

\subsubsection{Measurement Method}
    Measurements were taken directly from the power supply. More precise measurements pending.
\subsubsection{Measurement Result}

\section{Testing Protocol}
\subsection{Functionality}
\subsection{Noise}
\subsection{Software Tests}

\section{Verification}
    \subsection{Telduino}
        \subsubsection{Inspection}
            Solder bridges on the atmega micro-controller
            Solder bridges or incomplete connections on the FTDI usb-serial converter
            Reversed polarity of capacitors see layout for correct polarity
                For power circuit the output voltage may slowly drop if any of the polarized capacitors are in reverse and you are using the linear regulator.
                This is also true of the polarized capacitors near the modem and the atmega.
                If an AC/DC adapter is used without the ability to detect a short, the capacitors may ignite.
                The capacitors near the crystal are very close.

        \subsubsection{Testing}
            * Ensure jumpers are disconnected
                 Isolate Modem
                 -- Power: MDMPWER JP4 disconnects VCC from modem
                 -- RT TX MDMuCSer disconnects modem control from uC
                 -- JMDMSW disconnects On/Off modem control form uC
                 -- JRST disconnects modem hardware reset from uC
                 
                 Isolate Power from entire board
                 -- JP5 disconnects the power module from everything else.

            Test
            Expected
            Error
            GSM
            Test for GSM serial communications
            Type AT at the terminal will recieve OK back.
            Test antenna basic functionality
            Send a quick SMS using the commands
            Expected to recieve in x seconds

            Test correct power voltage
            Expect 3.6V across GND and JP5
            Expect 5V input across GND and 5VTST

            ***  Maximum load is 2A put X ohm resistor across pins A and B and ensure that 3.6V is still present

            Test USB to serial chip
            Connect to computer via USB
            Ensure connection to computer with commands:
                dmesg after connecting ensure no errors. On linux expect:
             
                usb 6-1: new full speed USB device number 4 using uhci_hcd
                ftdi_sio 6-1:1.0: FTDI USB Serial Device converter detected
                usb 6-1: Detected FT232RL
                usb 6-1: Number of endpoints 2
                usb 6-1: Endpoint 1 MaxPacketSize 64
                usb 6-1: Endpoint 2 MaxPacketSize 64
                usb 6-1: Setting MaxPacketSize 64
                usb 6-1: FTDI USB Serial Device converter now attached to ttyUSB0

            Leaving the rest of the board unpowered
            Connect pins OK3.4 to OK4.2 and type on the terminal, you should see an echo

            jumped jp5
            smoke test
            3.6V expected across JP5
            

            Test Micro Controller
                Use JTAG programmer as primary programmer. Use ICSP as backup
                
                Programming code
                

                You should see the following messages
                **********************************************
                vrdude: verifying ...
                avrdude: 39290 bytes of flash verified

                avrdude: safemode: Fuses OK

                avrdude done.  Thank you.

                Program fuses
                The following messages are expected 
                ******************************
                avrdude: verifying ...
avrdude: 1 bytes of efuse verified

avrdude: safemode: Fuses OK

avrdude done.  Thank you.


Comm Test to mirco
    Connect USB to board
    Reset micro

    The follow message is expected
        ****************************************
        telduino power up
        last compilation
        Mar 19 2012
        23:41:08
        ****************************

    At prompt enter 'a' 

    Expected Response: "Enter anme of register to read:"
    Enter "IRMS"
    Expected response: "regData:SUCCESS: hex value: decimal

Jumper Modem jumpers: 

Smoke test

    Expected current draw with modem off 150mA
    
Manual powering modem on

    Expected current draw is ~250mA

Comm test
First enter "at"
Expected Response: "OK"
Test SIMCard
    To test simcard connection:
    Enter "AT+CIMI"
    Expected IMSI from sim card a several digit number.


        Test< Power Draw
                Without modem connected: ~50mA at 5V
                With modem connected and powered off ~150mA
                With modem on ~250mA



    \subsection{Meter Daughterboard}


            
        \subsubsection{Inspection}
            The DC/DC converter can easily be unseated and put out of alignment. Check the leads for colds solders.
            The Optoisolators can be easily cold soldered as well.
            Check the crystals around the ADEs for solder bridges. The capacitors are very close.
        \subsubsection{Testing}
            Test voltage across gnd and vcc test points should be between 5 and 5.5
            Voltage across C32 should be input voltage 5V

            Expected Power Draw: 100mA at 5V

    \subsection{System}
    \subsubsection{Noise}
        Real world loads
        CFL
        Fan
        Power charger
        High frequency Switching test

    \subsubsection{Metering Capability}
        Test small load
        Test minimum visible
        Test minimum measureable
        Test maximum
        Test large load
        Test reactive load
        Test passive load
        Test long duration low frequency switch and meter functionality
        Test GSM transmission during metering
    \subsubsection{Ratings}
        Maximum current draw
            A large 1kW hotplate or two floodlights will easily draw 10A (1kW at 120V).
        Maximum voltage
            Maximum rated voltage is 250VAC. 
            An inverter or step-up converter will do the trick.
    \subsection{Software Overview}
    \subsection{ToFix}
        See: https://github.com/modilabs/telduino/issues/
\end{document}

TODO add female headers to parts list for completeness
Why is the Z test routine having issues?
Test Power draw at 2A
Replace buffer with level translator, right now the modem is mostly protected, but the uC can possibly output higher voltage than the level transaltors limit of 2.8=VAUX

